\chapter{Analytical evaluation of second-order corrections}
\addtocontents{toc}{\contentsline{chapter}{Analytical evaluation of second-order corrections}{\protect\pageref{annotation}}}
\label{ch:IntGreen}

Formulas presented in the previous section may be sufficient for numerical evaluation of the second-order corrections, provided one is careful in avoiding the numerical instabilities in evaluating the limit in~\eqref{Glimit}. However, since our main aim is the derivation of analytical approximations, we need to investigate ways of performing the radial integration analytically. 

\section{Closed form of the reduced Coulomb Green's function}

Following Johnson and Hirschfelder~\cite{JandH} we turn the limit in~\eqref{Glimit} into a derivative
\begin{align}
    \tilde{G}_n &= \lim_{\delta \rightarrow 0} G_{E_n+i \delta} - \frac{|\psi_n \rangle \langle \psi_n|}{i \delta}
    %\nonumber \\
    %&= \left[G_E-E_n\partial_E G_E\right|_{E=E_n} = \left[\partial_E (E-E_n)G_E\right|_{E=E_n}.
    = \left[\partial_E (E-E_n)G_E\right|_{E=E_n}.
    \label{Gderiv}
\end{align}
Since the hydrogen Green's function can be expressed in terms of hydrogen bound states according to~\eqref{GreenESch} and~\eqref{GreenEDirac}, and the energy is only dependent on the principal quantum number $n$, all that is required are the derivatives of the functions $R_{n,l,Z}(r)$ and $U_{n,l,Z}(r)$ over the first parameter. Differentiating the corresponding infinite series representation\footnote{A general series representation can be obtained by using~\eqref{WhittakerDefEq} to expand~\eqref{generalR}.} gives
\begin{align}
\partial_{\nu} R_{\nu,l,Z}(r)|_{\nu=n} =& \sqrt{Z(n+l)!(n-l-1)!}2\frac{(-1)^l}{n^2 \lambda} \nonumber
\\
\times&\Bigg[(-1)^l\sum_{i=-l}^l(\lambda)^i \frac{(l-i)!}{(l+i)!}\left(e^{\lambda/2}\frac{(n+i-1)!}{(n+l)!(n-l-1)!} -\frac{e^{-\lambda/2}}{(n-i)!}\right) \nonumber 
\\
&+\sum_{i=l+1}^n(-\lambda)^i\left(e^{\lambda/2} B_i- e^{-\lambda/2} \frac{A_i+\frac{\lambda}{2} C_i+\log(\lambda)-\rm{Ei}(\lambda)}{(l+j)!(n-j)!}\right)\Bigg],
\\
\partial_{\nu} U_{\nu,l,Z}(r)|_{\nu=n} =& -\sqrt{Z(n+l)!(n-l-1)!}2\frac{e^{-\lambda/2}}{n^2 \lambda}
\Bigg[\sum_{i=-l}^l(\lambda)^i \frac{(l-i)!}{(n-i)!(l+i)!} \nonumber 
\\
&\mspace{100mu}-(-1)^l\sum_{i=l+1}^n(-\lambda)^i\frac{A_i+\frac{\lambda}{2} C_i+\log(\lambda)}{(l+j)!(n-j)!}\Bigg],
\end{align}
where $\lambda=2Zr/n$ and we have defined the numerical coefficients:
\begin{align}
  A_i &=\frac{\Psi(n-l)+\Psi(n+l+1)}{2}-\Psi(i-l)-\Psi(l+i+1)-\frac{l+2}{n},
  \\
  B_i &= \sum_{j=i}^{n-1}\frac{(-1)^{j+i} (j-i)!}{(j-l)!(j+l+1)!(n-j-1)!},
  \\
  C_i &= \frac{1-i+l+2n}{n(i+l+1)},
\end{align}
using the polygamma function $\Psi(x)$, defined as
\begin{equation}
    \Psi(x) = \partial_x \log(\Gamma(x)).
\end{equation}

\begin{table}[b]
      \centering
    \begin{tabular}{cccc|ccc}
         $n$ & $l$ & $q$ & $q'$ & Analytic time & Numeric time & $\delta$\\
        \hline
        \hline
    3 & 1 & 2 & 0 & 8$\times 10^{-3}$ s & 3.83 s & $10^{-4}$\\
    7 & 5 & 4 & 1 & 0.128 s & 2.39 s & $10^{-4}$\\
    16 & 15 & 10 & 10 & 0.188 s & NaN & NaN\\
    37 & 1 & 1 & 0 & 8.38 s & 39.6 s & $10^{-7}$\\
    \hline
    \end{tabular}
  \caption{Comparison of computational times of the generating
    integral of the RCGF given by~\eqref{GnSchr} for different values of the parameters $n$, $l$, $q$ and
    $q'$ of analytical expressions~\cite{Dzikowski_2020} with the direct numerical integration using Mathematica~\cite{Mathematica}. The last column shows the value of $\delta$
    required to obtain results accurate to four significant
    figures. NaN means that the integral did not converge to the
    correct value for any value of $\delta$.}\label{tab:GIntTime}
\end{table}

Plugging those derivatives into~\eqref{Gderiv} produces a closed form of the RCGF
\begin{equation}
  G_{nl}(r_1,r_2) = \frac{4Z}{n} \frac{(n-l-1)!}{(n+l)!}
  \left(G_{nl}^{(\mathrm{sg})}(r_1,r_2) +
  G_{nl}^{(\mathrm{nsg})}(r_1,r_2)\right), \label{eq:68}
\end{equation}
where $G^{(\mathrm{sg})}_{nl}$ contains all terms that are singular
around the origin and $G^{(\mathrm{nsg})}_{nl}$ those that are not
\begin{align}\label{GnSchr}
  G_{nl}^{(\mathrm{sg})}(r_1,r_2)
  = &(-1)^{l}e^{\frac{-\lambda_1-\lambda_2}{2}} \sum_{i_1=l}^{n-1} \sum_{i_2=1-l}^{1+l}
    \beta{i_1}  \frac{(l+i_2-1)!}{(l+1-i_2)!}\nonumber
  \\
  &\times \left[\frac{(n-i_2)!}{(n-l-1)!}e^{\lambda_<}
    \frac{\lambda_>^{i_1}}{\lambda_<^{i_2}} - \frac{(n+l)!}{(n-1+i_2)!}
    \left(\frac{\lambda_1^{i_1}}{\lambda_2^{i_2}} + \frac{\lambda_2^{i_1}}{\lambda_1^{i_2}} \right)
    \right],
  \\
  G_{nl}^{(\mathrm{nsg})}(r_1,r_2)
  = -&e^{\frac{-\lambda_1-\lambda_2}{2}}\sum_{i_1=l}^{n-1} \sum_{i_2=l}^{n-1}
    \beta_{i_1}\beta{i_2}
    \Big[(i_2-l)!(i_2+l+1)!B_{i_2}\lambda_>^{i_1}\lambda_<^{i_2}e^{\lambda_<}
     \nonumber
  \\
  &- \lambda_1^{i_1} \lambda_2^{i_2} \left(\log\left(\lambda_>\right) - \rm{Ein}\left(\lambda_<\right) + A_{i_1}(\lambda_1)+A_{i_2}(\lambda_2)+C \right)\Big],
\end{align}
where $\lambda=2Zr/n$, Ein(x) is the Einstein function (see Appendix~\ref{app:Ein}) and we have defined:
\begin{align}
  A_{i}(x) &= \frac{2n+l-i}{i+l+2}\frac{x}{2n}- \Psi(1+i-l)- \Psi(2+i+l), \label{eq:21}
  \\
  B_{i}&=\sum_{k=1}^{n-i-1}
    \frac{(n-i-1)!(-1)^{k}(k-1)!}
    {(k+i-l)!(k+i+l+1)!(n-i-k-1)!},
    \\
    C&=-\frac{4l+5}{2n} + \Psi(n+l+1) + \Psi(n-l)-\gamma,
    \\
    \beta_i &= (-1)^i\frac{(n+l)!}{(n-i-1)!(l+i+1)!(i-l)!},
\end{align}
where the Euler-Mascheroni constant is $\gamma \approx 0.57722$.
As expected this is equivalent to the form originally found by Johnson and Hirschfelder~\cite{JandH}.

\begin{table}[b]
       \centering
    \begin{tabular}{ccc|ccc}
         $n$ & $l$ & $q$ & Analytic time & Numeric time & $\delta$\\
        \hline
        \hline
    1 & 0 & 0 & 0.19 s & 21.7 s & $10^{-4}$ \\
    2 & 4 & 3 & 0.01 s & 32.2 s & $10^{-4}$ \\
    5 & 4 & 7 &  3.89 s & 68.1 s & $10^{-5}$ \\
    6 & 8 & 8 &  0.05 s & 53.5 s & $10^{-6}$ \\
    \hline  \end{tabular}
  \caption{Comparison of computational times of the integral moments~\eqref{Gmoment}
    for different values of parameters $n$, $l$, $q$ and $\lambda$ of
    analytical expressions from~\cite{Dzikowski_2020} with the direct numerical integration of  using
    Mathematica~\cite{Mathematica} at 100 different values of $x$.
    The last column shows the value of $\delta$ required to obtain the
    results accurate to four significant figures.}\label{tab:2}
\end{table}

\section{Integrating the Reduced Coulomb Green's function}

The generating integral of the RCGF
\begin{equation}\label{GenerateG}
    K_{nl}(\lambda, \lambda') = \int e^{-\lambda x - \lambda' x'} G_{nl}(x,x')x^{q} x'^{q'} dx dx',
\end{equation}
is convergent for $q \geq 0$ and $q' \geq 0$. The main difficulty in performing such integration analytically is the fact, that integrals of individual terms in~\eqref{GnSchr} may not converge, even though the full integral~\eqref{GenerateG} converges overall. For this reason, the strategy
employed in our work~\cite{Dzikowski_2020} was to first derive expressions valid for
non-integer values of $q$ and $q'$, then find the Laurent series of
\eqref{GenerateG} around $q = m + \delta$,
$q' = m' + \delta$, where $m$ and $m'$ are non-negative integers and
finally show that for $q,q'\geq0$ the divergent parts (terms
proportional to $\delta^{-1}$ and $\delta^{-2}$) always vanish.

This is done most conveniently by expressing the integral in terms of the generating integrals of the Heaviside step function
\begin{align}
   u_{a}^{b}(x,y) =&\int_0^\infty \int_0^\infty e^{-\lambda r - \lambda' r'} r^{a-1}
  {r'}^{b-1} \theta(r'-r)dr' dr \nonumber
  \\
  &= \int_0^\infty \int_r^\infty e^{-\lambda r - \lambda' r'} r^{a-1}
  {r'}^{b-1} dr' dr,
\end{align}
convergent whenever $\text{Re}(a)>0$, $\text{Re}(a + b) >0$ and
$\text{Re}(\lambda + \lambda') > 0$. In general it is given by
\begin{align} 
  u_{a}^{b}(x,y) =
  \sum_{i=0}^\infty \frac{\Gamma(a+b+i)}{(a+i)x^{a+b+i}}
  \frac{(-y)^i}{i!}. \label{uGen}
\end{align}

In the case when $a$ and $b$ are positive integers it simplifies to a finite sum
\begin{align} 
  u_{a}^{b} (x,y) = \frac{\Gamma(a)}{y^a} \left(\frac{\Gamma(b)}{x^b} -
  \left(x+y\right)^{-b} \sum_{i=0}^{a-1} \frac{\Gamma(b+i)}{i!}
  \left(\frac{y}{y+x}\right)^i\right).\label{uInteger}
\end{align}

Equations~\eqref{uGen} and~\eqref{uInteger} are convenient for deriving the Laurent series of $u$ and subsequently, the closed form of~\eqref{GenerateG}. The resulting expression is somewhat involved, so we avoid writing it out here, but it can be found along with all corresponding derivations in~\cite{Dzikowski_2020}.

In table~\ref{tab:GIntTime} we compare the evaluation times of
the generating integral for different values of parameters $n$, $l$,
$q$ and $q'$ between the analytical expression given in~\cite{Dzikowski_2020} and a direct numerical integration of~\eqref{GnSchr} with a fixed value of $\delta$. As can be observed from the table the numerical
evaluation is several orders of magnitude slower than our analytical
expressions. It is also clear that the evaluation time increases significantly when $n \gg l$. We have found out that in all relevant cases the total evaluation time through analytical expressions is of the order of 0.001-0.1 seconds on Intel 2600k 3.4GHz processor.

\begin{table}[b]
\centering
  \begin{tabular}{cc|ccc}
     Element & Atomic number & Analytic time & Numeric time & $\delta$\\
    \hline
    \hline
    Li & 3 & 3.26 s & 29.3 s & $10^{-3}$\\
    F & 9 & 14.2 s & 261 s & $10^{-4}$\\
    Ne & 10 &  29.3 s & 321 s & $10^{-4}$\\
    \hline 
  \end{tabular}
  \caption{Comparison of computational times of second order
    single-electron correction to ground state energies of some
    example neutral atoms. The last column shows the value of $\delta$ required to obtain the
    results accurate to four significant figures.} \label{tab:3}
\end{table}

 On the other hand, the direct numerical
integration of the Green's function for some finite values of $\delta$
becomes very inefficient for large values of $n$ and $l$. This happens due to the increasing number
of nodes of the integrand, resulting in oscillatory
behavior. Consequently, the accurate evaluation of the integral
demands the smaller and smaller values of $\delta$ to keep the
constant accuracy, since in many cases large positive values are
almost completely cancelled by large negative values. Therefore, the
precise result would require a forbiddingly accurate evaluation of the
integrand at every point. Consequently, the evaluation
time of some high-$n$ Rydberg states is still large and requires
further optimization, for example, by expanding the analytic results in an asymptotic series in $n$.

For the purpose of obtaining analytical corrections to hydrogen-like wavefunctions, one also needs to consider the integral moments of the RCGF
\begin{equation} \label{Gmoment}
    J_{nl}(\lambda, x') = \int e^{-\lambda x} G_{nl}(x,x')x^{q} dx.
\end{equation}
In table~\ref{tab:2} we compare the evaluation time of analytically
computed integral moments with the numerically computed ones for some
values of the parameters $q$, $n$, $l$ and $\lambda$. In this
simulation, we numerically evaluated the integral moments at one
hundred different values of $r$ in order to compare with an
analytically calculated curve. As in the situation of the generating
integral, the evaluation time of the direct numerical integration is a
few orders of magnitude slower.

We also
compare a few cases of the second-order single-electron correction evaluation times between our analytical and
numerical approaches. The result is given in table~\ref{tab:3}. Typically a
fully sequential version of an ECM program implemented in Mathematica
requires one order of magnitude larger times for the direct numerical
evaluation, as compared to the analytical expressions.

\section{The Reduced Dirac-Coulomb Green's function}

The RDCGF can be handled in an analogous way to the RCGF, since~\eqref{Gderiv} is equally valid in the relativistic case. If the relativistic Green's function is expressed in terms of the Dirac hydrogen wavefunctions according to~\eqref{GreenEDirac}, then all that is needed are the derivatives of the Dirac wavefunctions over the principal quantum number $n$. This can be achieved by expressing the Dirac wavefunctions in terms of the Schr\"odinger ones according to~\eqref{Diracwave}. The additional complication however, comes from the fact, that the relativistic calculation requires evaluating the derivatives of the $R_{n,l}$ and $U_{n,l}$ functions at non-integer values of $n$ and $l$, in which case they cannot be expressed as a finite series. In terms of the regularized hypergeometric functions ${}_p\widetilde{F}_q$, we can write:
\begin{align}
\partial_{\nu} R_{\nu,\gamma,Z}(r) &= \sqrt{Z q!\Gamma(p+1)}2\frac{e^{-\lambda/2}}{n^2 \lambda} \nonumber
\\
&\times\Bigg[\sum_{i=0}^{q}\frac{(-\lambda)^{i}}{i!}\frac{(\lambda)^{\gamma+1}A_i}{(q-i)!\Gamma(i+2\gamma+2)} +\frac{(-1)^qq}{(q+1)!}\frac{p}{\Gamma(p+2)}\frac{(\lambda)^{\nu+1}}{2\nu} \nonumber 
\\
&\mspace{180mu}-(-1)^q\lambda^{\nu+2}{}_2\widetilde{F}_2[1,2;q+3,p+3;\lambda]\Bigg], 
\\
\partial_{\nu} U_{\nu,\gamma,Z}(r) &= \partial_{\nu} R_{\nu,\gamma,Z}(r)+\sqrt{Z q!\Gamma(p+1)}\frac{2}{n^2 \lambda} \nonumber
\\
&\times\Bigg[e^{-\lambda/2}\sum_{i=0}^{q}\frac{(-\lambda)^{i}}{i!}\frac{(\lambda)^{\gamma+1}(\Psi(-1-p)-\Psi(p))}{(q-i)!\Gamma(i+2\gamma+2)} \nonumber 
\\
&\mspace{170mu}-(-1)^q e^{\lambda/2}\frac{\Gamma(-p)}{\lambda^{\gamma}}{}_1\widetilde{F}_1[q+1,-2\gamma,-\lambda]\Bigg],
\end{align}
where  and we have defined the coefficients:
\begin{align}
p=\nu+\gamma, \qquad \qquad q=\nu-\gamma-1, \qquad \qquad \lambda=2rZ/n,
\\
  A_i =\frac{\Psi(q)+\Psi(p)}{2}-\Psi(q-i)+\frac{i-1-2\nu}{2\nu}\frac{i}{q+1-i}-\frac{\gamma+2}{\nu}.
\end{align}
Note that these formulae assume that $q$ is a non-negative integer, but $p$ isn't, as is always the case when evaluating the Dirac hydrogen-like wavefunctions in terms of the Schr\"odinger ones (see~\eqref{Diracwave}).

The relativistic description also poses a more
complex dependence on effective charge. Due to the mass term of the
Dirac equation, it is impossible to separate
explicitly the dependence of the matrix elements on the effective
charge. For this reason, results of the D-ECM calculations presented in this thesis focus on the leading-order
approximation.

%\cite{wong_dirac_1985,swainson_unified_1991}

