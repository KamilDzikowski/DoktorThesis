\thispagestyle{empty}
\begin{otherlanguage}{ngerman}
{\small
  \begin{center}
    \textbf{Zusammenfassung}
  \end{center}

  \noindent Das k\"urzlich vorgeschlagene Effektivladungsmodell (ECM) ist ein neuartiger Ansatz, um vollst\"andig analytische Ann\"aherungen an die beobachtbaren Eigenschaften von Mehrelektronenatomen und -ionen zu erhalten. Es verwendet einen wasserstoff\"ahnlichen Basissatz mit einem einzigen Parameter, der als effektive Ladung bezeichnet wird, um eine St\"orungstheorie zu konstruieren. Die zugeh\"orige St\"orungsreihe konvergiert schnell, ber\"ucksichtigt auf nat\"urliche Weise Korrelationseffekte und erm\"oglicht eine effiziente Berechnung aller nachfolgenden Korrekturen. In dieser Arbeit vergleichen wir die Genauigkeit der durch das ECM erzeugten analytischen N\"aherungen mit Ergebnissen anderer gebr\"auchlicher Methoden, wie der Hartree-Fock-Methode und dem Thomas-Fermi-Modell, sowohl innerhalb der relativistischen als auch der nicht-relativistischen Quantenmechanik. Zu diesem Zweck werten wir Grundzustandsenergien, Energien angeregter Zust\"ande und Ionisationsenergien sowie eine Vielzahl anderer atomarer Eigenschaften wie Elektronendichten, Streufaktoren, Photoionisationsquerschnitte und \"Ubergangswahrscheinlichkeiten f\"ur ein breites Spektrum von Systemen,
von neutralen Atome bis zu hochgeladenen Ionen, aus. Wir zeigen auch, wie die Greensche Funktion des Wasserstoffatoms analytisch integriert werden kann, um eine effiziente Berechnung der ECM-Korrekturen zweiter Ordnung zu erm\"oglichen. Schlie\ss lich untersuchen wir verschiedene zus\"atzliche Effekte, die die Genauigkeit der ECM-N\"aherungen korrigieren, insbesondere solche, die durch die Breit-Wechselwirkung, durch endlichen-Kerngr\"o\ss e-Effekte und durch die Vakuumpolarisation hervorgerufen werden. Da die Genauigkeit der ECM-Approximationen zweiter Ordnung bereits mit Ergebnissen eines Hartree-Fock-Ansatzes mit mehreren Konfigurationen vergleichbar ist, k\"onnen wir uns vorstellen, dass das ECM das Thomas-Fermi-Modell, f\"ur alle Anwendungen, in denen es noch verwendet wird ersetzen kann.} %dass das ECM andere Modelle vergleichbarer Komplexit\"at, wie zum Beispiel das Thomas-Fermi-Modell, f\"ur alle Anwendungen, wo es noch genutzt wird ersetzen kann.}
\end{otherlanguage}

{\small
  \begin{center}
    \textbf{Abstract}
  \end{center}

  \noindent The recently proposed effective-charge model (ECM) is a novel approach to producing fully analytical approximations to the observable characteristics of multi-electron atoms and ions. It employs a hydrogen-like basis set with a single parameter, called effective charge, for the construction of perturbation theory. The associated perturbation series converges fast, includes correlation effects in a natural way and enables an efficient calculation of all subsequent corrections. This work compares the accuracy of the analytical approximations produced by the ECM to results of other commonly used methods, such as the Hartree-Fock method and the Thomas-Fermi model, within both relativistic and non-relativistic quantum mechanics. For this purpose, ground state, excited state and ionization energies are evaluated, as well as a wide range of other atomic characteristics, such as electronic densities, scattering factors, photoionization cross-sections and transition probabilities, for a wide range of systems, from neutral atoms to highly charged ions. It is also shown how the Green's function of the hydrogen atom can be integrated analytically, allowing for an efficient calculation of the second-order ECM corrections. Finally, various additional effects that correct the accuracy of the ECM approximations are investigated, in particular those originating from the Breit interaction, finite-nuclear-size effects and vacuum polarization. Given that the accuracy of the second-order ECM approximations is already comparable with results obtained using a multi-configuration Hartree-Fock approach, we envisage that the ECM can replace the Thomas-Fermi model for all applications where it is still utilized.}

%%% Local Variables:
%%% mode: latex
%%% TeX-master: "../doc"
%%% End:
